\documentclass[12pt]{article}
\usepackage[usenames]{color} %used for font color
\usepackage{graphicx}
\usepackage{amssymb} %maths
\usepackage{amsmath} %maths
\usepackage[utf8]{inputenc} %useful to type directly accentuated characters

% PDF Bookmarks and hyperref stuff
\usepackage[
  bookmarks=true,
  bookmarksnumbered=true,
  colorlinks=true,
  filecolor=blue,
  linkcolor=blue,
  urlcolor=blue,
  hyperfootnotes=true
]{hyperref}

\author{David MacMahon}
\title{VLBI Phased Array \\ iBob and BEE2 \\ Registers and Devices}

\begin{document}
\maketitle
\setlength{\parindent}{0pt}
\setlength{\parskip}{1ex plus 0.5ex minus 0.2ex}
\newlength{\itemseporig}
\setlength{\itemseporig}{\itemsep}

% Put new commands here
\newcommand{\tighten}{
  \setlength{\itemsep}{-\parskip}
}
\newcommand{\loosen}{
  \setlength{\itemsep}{\itemseporig}
}
\newcommand{\LSB}{least significant byte }
\newcommand{\LSb}{least significant bit }
\newcommand{\LSbs}{least significant bits }
\newcommand{\devitem}[2]{\item[#1] [#2] \\ }
\newcommand{\rwreg}[1]{\devitem{#1}{R/W register}}
\newcommand{\rwregn}[1]{\item[#1] {} \tighten} % For "stacked" rwregs
\newcommand{\rwregs}[1]{\devitem{#1}{R/W registers} \loosen} % Ends stacked rwregs
\newcommand{\roreg}[1]{\devitem{#1}{R register}}
\newcommand{\roregn}[1]{\item[#1] {} \tighten} % For "stacked" rregs
\newcommand{\roregs}[1]{\devitem{#1}{R registers} \loosen} % Ends stacked rregs
\newcommand{\degree}[1]{$#1^{\circ}$}

\begin{abstract}
Describes the registers and devices in the various iBob and BEE2 designs used
in the VLBI phased array processor for CARMA and SMA.
\end{abstract}

\section{ibob\_phased\_array\_2k}

The ibob\_phased\_array\_2k design has a few critical registers that must be
set to non-zero values for proper operation.  These registers are briefly
listed here; full details are provided below.

\begin{description}
\rwreg{valid} Must be 1 to enable data to flow through the delay line.
\rwreg{start\_xaui} Must be 1 to enable XAUI transmission as well as the clock
output for the BEE2 calibration correlator.
\rwregs{gain0 - gain3} At least one gain register must be non-zero to get
non-zero output.
\end{description}

  \subsection{Sync and 1~PPS}
To facilitate synchronized seeding of noise sources on different devices, the
noise generators are armed via software.  This causes them to re-seed on the
next rising edge of the internal 1~PPS signal.  The internal 1~PPS is generated
by counting clock cycles and can be synchronized with one or the other of the
ADCs sync pulses.

\begin{description}
\rwreg{ppssel} Selects the ADC sync pulse to which the interally generated
1~PPS signal will be synchronized.

0: Select ADC0 sync signal (typically for CARMA) \\
1: Select ADC1 sync signal (typically for SMA)

\rwreg{onepps\_sync/arm1pps} A 0 to 1 transition of the \LSb of this register
will arm the internal 1~PPS generator to re-synchronize with the next rising
edge of the selected sync pulse.

\roreg{onepps\_sync/arm1pps\_ack} A 1 to 0 transition of the \LSb of
\verb|onepps_sync/arm1pps| clears (i.e. sets to zero) this register.  This
register is set to 1 on the rising edge of the next sync pulse that
re-synchronizes the internal 1~PPS generator.  This can be used to verify that
the internal 1~PPS generator did indeed re-synchronize after being armed.

\end{description}

  \subsection{Input Selection}

\begin{description}

\rwreg{insel}  Input selection.  Uses one bit per demux path per input.

0: Select ADC \\
1: Select noise source

Examples:
\begin{description}

\item[0x0000] Select ADC for all demux paths of all inputs.
\item[0x0001] Select noise source for demux path 0 of input0, select ADC  for
               all other demux paths.
\item[0x000F] Select noise source for all demux paths of input 0, select ADC
               for all other inputs.
\item[0xFFFF] Select noise source for all demux paths of all inputs.
\item[0xF000] Select noise source for all demux paths of input 3, select ADC
               for all other inputs.
\item[0xC000] Select noise source for demux paths 2 and 3 of input 3, select
               ADC for all other inputs.
\end{description}
\end{description}

  \subsection{Noise Generators}

\begin{description}

\rwreg{noise/arm}  A 0 to 1 transition of the \LSb of each
nybble arms that input's noise generator.  The least significant nybble
corresponds to input0.  A value of 0x0011 wil arm the noise generators for
input0 and input1.  A value of 0x1111 will arm all noise generators.

Once a noise generator is armed, it will re-seed starting on the next internal
1~PPS rising edge.  This can be used to synchronously seed noise generators
across multiple devices or within one device.

\filbreak
\rwregn{noise/seed/0}
\rwregn{noise/seed/1}
\rwregn{noise/seed/2}
\rwregs{noise/seed/3} Sets 32~bit seed for noise generators for corresponding
inputs.  Because the inputs are demultiplexed by 4, each noise generator
actually contains two noise generators that each output two indpendent normally
distributed values each.  The 32~bit seed is actually split into two 16~bits
seeds to seed the two contained noise generators.  Using a seed whose two
halves are identical (i.e.\ a multiple of 65537) will result in a signal that
has a lag-2 correlation coefficient of 1 resulting in a non-flat spectrum.

\end{description}

  \subsection{XAUI}
The sum of the phased inputs is sent to the DBE via the XAUI0 connector.  The
2-bit quantized inputs are sent to the calibration correlator via the XAUI1
connector.  XAUI data transmission is \emph{dis}abled by default and must be
explicitly enabled via software (see below).

\begin{description}

\rwreg{start\_xaui} Writing a 1 to this register enables data transmission over
both XAUI links as well as enabling the dive-by-two counter used to drive the
BEE2 calibration correlator's \verb|usr2_clk|.

Writing a 0 (the default) disables XAUI data transmission over both XAUI links
as well as enabling the dive-by-two counter used to drive the BEE2 calibration
correlator's \verb|usr2_clk|.

\filbreak
\roregn{xaui\_0/rx\_linkdown}
\roregs{xaui\_1/rx\_linkdown} A value of 1 indicates a ``link down'' status on
the (unused) receive side of the indicated XAUI connection, which strongly
implies a similar status for its transmit side.

\end{description}

  \subsection{Front Panel SMA Output}
\begin{description}

\rwreg{smasel} Selects the signal output over the front panel \verb|SMA0|
connector.

\begin{description}

\item[0] Outputs a 128 MHz square wave (50\% duty cycle) if XAUI ouput is
enabled, otherwise a constant 0 or 1.

\item[1] Outputs a slightly delayed copy of ADC0's sync
input.

\item[2] Outputs the SMA walsh signal used for input1.

\end{description}
\end{description}

  \subsection{Fractional Delay FIR Filter}

\begin{description}

\rwreg{valid} Writing a 1 to this register enables the delay line that performs
coarse (FPGA clock cyles) and fine (ADC clock cycles) delays before feeding the
fractional (i.e. sub-sample) delay filter.  Writing a 0 (the default) causes
the delay line to output a constant value from some earlier sample.  When
\verb|valid| is set, LED1 will be lit.

\filbreak
\rwregn{delay1}
\rwregn{delay2}
\rwregn{delay3}
\rwregs{delay4} Sets the coarse delay in units of $4 T_s$.  Only the 11 \LSbs
are used corresponding to a coarse delay range of $1 T_s$ to $8192 T_s$ (a 0
value is effectively 2048).  Note that the \verb|delayN| registers are numbered
1 through 4 corresponding to inputs 0 through 3.

\filbreak
\rwregn{select1}
\rwregn{select2}
\rwregn{select3}
\rwregs{select4} Sets the fine delay in units of $T_s$.  Only the 2 \LSbs
are used corresponding to a coarse delay range of $0 T_s$ to $3 T_s$.  Note
that the \verb|selectN| registers are numbered 1 through 4 corresponding to
inputs 0 through 3.

\filbreak
\rwregn{fdelay1}
\rwregn{fdelay2}
\rwregn{fdelay3}
\rwregs{fdelay4} Sets the sub-sample delay in units of $\frac{1}{10}T_s$.  Only
the 4 \LSbs are used, but the valid sub-sample delay range is
$\frac{0}{10} T_s$ to $\frac{9}{10} T_s$.  Note that the \verb|fdelayN|
registers are numbered 1 through 4 corresponding to inputs 0 through 3.

\filbreak
\rwregn{bypass1}
\rwregn{bypass2}
\rwregn{bypass3}
\rwregs{bypass4} A value of 1 will bypass the sub-sample delay FIR filter
altogther (but not the coarse or fine delays).  Although bypassing is
theoretically equivalent to setting \verb|fdelayN| to 0, it differs in that the
non-theoretical filter passband is not imposed on the signal.  It also differs
in that the latency of the signal path is change (which could be considered a
bug).

\end{description}

  \subsection{Front Panel LEDs}
\begin{description}
\item[LED0] On when the \verb|valid| register is set to 15 (0xF).
\item[LED1] Not used.
\item[LED2] Briefly blinks on each rising edge of the sync0 input.
\item[LED3] Not used.
\item[LED4] Indicates the state of \degree{180} phase switching for input 0.
\item[LED5] Indicates the state of \degree{180} phase switching for input 1.
\item[LED6] Indicates the state of \degree{180} phase switching for input 2.
\item[LED7] Indicates the state of \degree{180} phase switching for input 3.
\item[LED8] Indicates FPGA is configured (non-functional on some iBobs).
\end{description}

  \subsection{TODO}
count\_out
datasel

\begin{description}
\filbreak
\rwregn{gain0}
\rwregn{gain1}
\rwregn{gain2}
\rwregs{gain3}
\end{description}

ibob\_lwip/ethlite
ibob\_lwip/macbits

reset

sum\_out

\begin{description}
\filbreak
\rwregn{thresh0}
\rwregn{thresh1}
\rwregn{thresh2}
\rwregs{thresh3}
\end{description}

wgrp\_sel
xcapt\_en
xcapt\_rst
xhb\_walsh

\section{dbe\_xaui\_corr}

The dbe\_xaui\_corr design receives two signals (i.e. streams of data) from two
different ibobs via XAUI.  It must time-align these two incoming signals, sum
them, and then process and format the final summed signal for recording to the
Mark 5 data recorder.  The design also contains a digital noise generator that
can be used in place of the final sum for engineering purposes.

\subsection{XAUI alignment and monitoring}

The alignment of the two incoming XAUI streams is critical to achieveing proper
summing.  The design will automatically align the two XAUI streams and, if ever
necessary, re-align them at the next 1024 PPS.  This ensures that the two
streams are never misaligned for more than 1/1024 of a second.  Additionally,
the out-of-band sync signals sent over the xaui link are monitored for any
anomalous behavior.  The registers involved in the XAUI alignment and
monitoring are described here.

\begin{description}
\rwreg{xaui\_rst}  Resets various parts of XAUI related logic.  Setting the
\LSb (bit 0) to 1 will reset the XAUI core itself (rarely needed).  Setting bit
1 to 1 will reset the counters associated with XAUI monitoring.  Setting bit 2
to 1 will reset the registers that capture the relative delays of the two XAUI
links.
\end{description}

xaui0/linkdown\_cnt
xaui0/period
xaui0/period\_err
xaui0/period\_err\_cnt
xaui0/sync\_cnt

xaui0/almost\_full
xaui0/rx\_empty\_cnt
xaui0/rx\_linkdown
xaui0/valid
xaui1/almost\_full
xaui1/linkdown\_cnt
xaui1/period
xaui1/period\_err
xaui1/period\_err\_cnt
xaui1/rx\_empty\_cnt
xaui1/rx\_linkdown
xaui1/sync\_cnt
xaui1/valid

correlate
delay\_adj
ibob\_lwip/ethlite
ibob\_lwip/macbits
insel
noise/arm
noise/seed
onepps\_sync/arm1pps
onepps\_sync/arm1pps\_ack
outsel
pol0/gainctrl0
pol0/gainctrl1
pol0/gainreset
shiftctrl\_reg
sma1pps\_sel
smavsisel
snap/addr
snap/bram
snap/ctrl
trig\_capt
xaui\_summer/corr\_out-4
xaui\_summer/corr\_out-8
xaui\_summer/corr\_out0
xaui\_summer/corr\_out4
xaui\_summer/corr\_out8
xaui\_summer/delay0
xaui\_summer/delay1


\section{dbe\_adc\_single}

\section{bee2\_calib\_corr}


\end{document}
